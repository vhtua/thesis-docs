\section{Internship Training Phases}
My internship period is segmented into two primary phases: the \textbf{Training Phase} and the \textbf{Internal Project Phase}. This section will provide a detailed discussion of the Training Phase.

\subsection{Overview}
    % The training schedule can be seen clearly as the figure below:
    % \begin{figure}[H]
    %         \centering
    %         \includegraphics[scale=0.5]{graphics/trainning-plan.png}
    %         \caption{Training schedule}
    %         \label{fig:trainingPlan}
    % \end{figure}
    At the beginning of the internship, I was assigned to train my technical skills directly under the supervision of my manager. This training phase spans two months and is designed to establish a robust technical foundation for all interns (from theories to practical programming). Each week during this period is dedicated to the completion of a specific topic.
    

\subsection{Training contents}
    \subsubsection{Basic C++, CMake, Git}
    As a computer science student currently training my technical skills at a leading company, my exploration of learning basic C++ programming has been both fundamental and transformative. Initially, I engaged with the core concepts of C++, including syntax, data types, and control structures, laying a solid groundwork for my programming skills. Through hands-on coding exercises, I developed a practical understanding of fundamental constructs such as loops, conditionals, and functions. As I progressed, I delved into more complex topics like memory management, pointers, and object-oriented programming principles, which provided deeper insights into the efficiency and power of C++. After establishing a solid foundation in C++ programming, I learned to use CMake for building, testing, and packaging projects. This experience improved my ability to manage complex build configurations and cross-platform development efficiently. \\
    
    \noindent Moreover, I acquainted myself with the basic concepts of version control, understanding the importance of tracking changes and maintaining code integrity. I progressed to mastering essential GIT commands such as \textit{clone}, \textit{commit}, \textit{push}, \textit{pull}, and \textit{merge}, enabling me to manage and synchronize code effectively across different repositories. Through practical exercises and real-world projects, I developed proficiency in branching strategies, resolving merge conflicts, and maintaining a clean and organized codebase. This comprehensive training has equipped me with the skills necessary to collaborate efficiently in a team environment, ensuring seamless version control and continuous integration within the software development lifecycle.

    \subsubsection{C++ Programming}
    In this segment, I delve further into the realm of C++ programming, acquiring proficiency in modern C++ Object-Oriented Programming (OOP) principles and mastering Standard Template Library (STL) usage. Subsequently, I attained a comprehensive comprehension of C++ programming, culminating in a succinct summary of my acquired knowledge detailed below:
    
    \begin{itemize}
            \item \textbf{Structure of a C++ Program:} comprises multiple source code files, each addressing distinct components such as the main function, member functions, class definitions, headers/standard headers, comments, variables, data types, namespaces, and input/output statements. \cite{structure_cpp}
            
            \item \textbf{Variables and Constants:} serve as fundamental elements for data storage and manipulation. Variables are named storage locations in memory that can hold different values throughout the program's execution. They are defined with specific data types, such as int, float, or char, which determine the kind of data they can store. Constants, on the other hand, are immutable values that, once defined, cannot be altered during the program’s runtime. Constants are declared using the const keyword, ensuring data integrity and preventing accidental modifications. Together, variables and constants facilitate efficient data handling and enhance the reliability of C++ programs.
            
            \item \textbf{Statements and Operators:} are integral to the language's functionality and control flow. Statements, such as expressions, declarations, and control structures (e.g., if, for, while), represent individual instructions that the compiler executes sequentially. Operators, including arithmetic (e.g., +, -, *), relational (e.g., ==, !=), logical (e.g., \&\&, ||), and assignment (e.g., =, +=, -=) operators, perform operations on variables and values within these statements. They enable the manipulation and evaluation of data, allowing developers to construct complex expressions and control program logic effectively. The interplay between statements and operators is foundational to writing coherent and functional C++ code.
            
            \item \textbf{Controlling Program Flow:} is achieved through constructs like conditional statements (if, switch) and loops (for, while, do-while). These constructs enable the execution of code based on specific conditions or repetition of code blocks, facilitating dynamic decision-making and iterative processes essential for efficient programming.
            
            \item \textbf{Functions:} are modular blocks of code designed to perform specific tasks. Defined with a return type, name, parameters, and body, functions enhance code reusability, readability, and maintainability. They allow for structured programming by enabling the division of complex problems into manageable sub-tasks, facilitating efficient program development and debugging.

            \item \textbf{Pointers and References:} are crucial features for managing memory and facilitating efficient data manipulation. Pointers store memory addresses, enabling direct access and modification of data. References provide an alias to an existing variable, simplifying code and avoiding unnecessary memory overhead. They both contribute to dynamic memory allocation and advanced data structures implementation.
            
            \item \textbf{Smart Pointers:} are dynamic memory management objects that automate memory allocation and deallocation, mitigating common issues like memory leaks. They ensure proper resource handling by encapsulating raw pointers and providing features like automatic memory deallocation when objects go out of scope. Smart pointers, including unique\_ptr, shared\_ptr, and weak\_ptr, enhance code safety and readability while promoting efficient memory management practices.
            
            \item \textbf{Object-Oriented Programming (OOP) Concepts:} form the foundation of software development, facilitating modular and scalable code structures. Encapsulation allows data hiding within objects, ensuring data integrity and security. Inheritance enables code reuse and hierarchy establishment, promoting extensibility and flexibility. Polymorphism fosters code abstraction and flexibility through methods with multiple implementations. These principles collectively enhance code organization, maintainability, and extensibility in C++ programs.
            
            \item \textbf{Exception Handling:} is a vital mechanism for managing errors and exceptional situations within programs. It provides a structured approach to deal with runtime errors, ensuring robustness and reliability. Through try, catch, and throw blocks, C++ programmers can segregate normal program flow from error-handling code, promoting code clarity and maintainability. Exception handling enhances program resilience by enabling graceful recovery from unexpected events, thereby contributing to overall software stability and quality.
            
            \item \textbf{Standard Template Library (STL):} is a comprehensive collection of reusable classes and functions, offering a rich set of data structures and algorithms. It comprises containers like vectors, lists, and maps, facilitating efficient data organization and manipulation. Additionally, STL provides algorithms for sorting, searching, and modifying data, enhancing code efficiency and productivity. Its modular design and wide adoption make it an essential component for developing robust and scalable C++ applications.
            
            \item \textbf{Lambda Expressions:} (was first introduced in C++11) provide a concise and flexible means of defining anonymous functions inline within code. They facilitate the creation of functional-style programming constructs, enabling developers to write more expressive and readable code. Lambda expressions offer syntactic sugar for defining closures, capturing variables from the enclosing scope, and facilitating algorithm customization. This feature enhances code clarity, promotes modularity, and contributes to the overall efficiency of C++ programs.
            
        \end{itemize}

    \subsubsection{Linux Programming, Shell Scripting}
    
    My journey into learning Linux Programming has been both challenging and rewarding. Initially, I familiarized myself with basic Linux commands, file system navigation, and process management, laying a strong foundation for further exploration. Delving deeper, I ventured into system programming and socket programming, honing my skills through practical application and experimentation. \\

    \noindent Concurrently, my progression in learning Shell Scripting has been characterized by gradual mastery of shell syntax, variable manipulation, and control structures. Starting with fundamental concepts, I steadily advanced to more intricate scripting techniques and best practices. Practical exercises and real-world scenarios provided invaluable opportunities to automate system tasks, manage file systems, and implement system administration tasks efficiently. Through persistent practice, experimentation, and exposure to diverse scripting challenges, I have developed the ability to craft robust, maintainable shell scripts tailored to Unix/Linux environments


    \subsubsection{Testing Concepts, Clean Code}
     Beginning with an exploration of fundamental testing concepts such as unit testing, integration testing, and regression testing, I have progressively advanced to more sophisticated techniques like test-driven development (TDD) and behaviour-driven development (BDD). Through exercises and hands-on experience, I have gained insights into test planning, execution, and automation. \\

     \noindent In parallel, my endeavours in mastering Clean Code have been characterized by a meticulous study of software engineering best practices aimed at fostering code readability, maintainability, and scalability. By immersing myself in the principles advocated by renowned industry experts and seminal works such as Robert C. Martin's "Clean Code," I have internalized the importance of writing concise, well-structured, and self-explanatory code. Through rigorous code reviews, refactoring exercises, and adherence to established coding conventions, I have cultivated a disciplined approach to software development, producing code that is not only functional but also elegant and easy to comprehend.


    \subsubsection{Software Design}
    During my training phase at BGSW Vietnam, my understanding of software design has significantly deepened. Initially, I had a theoretical grasp of design principles such as modularity, encapsulation, and the SOLID principles. However, real-world application has enriched my comprehension. Collaborating with experienced developers, I have learned to navigate complex codebases and apply design patterns like Singleton, Factory, and Observer effectively. This hands-on experience has highlighted the importance of designing for maintainability and scalability, ensuring that the software can adapt to changing requirements and handle increasing loads. Through code reviews and iterative development cycles, I have honed my skills in creating clean, readable, and efficient code. Additionally, I have gained proficiency in using UML diagrams for planning and communicating design ideas, fostering a collaborative environment where ideas are refined before implementation. This practical exposure has transformed my theoretical knowledge into actionable skills, preparing me for future challenges in software design.


    \subsubsection{Python, Flask framework}
     In the beginning, my Python knowledge was confined to basic scripting and data manipulation. Through structured projects and real-world tasks, I have advanced my understanding of Python’s robust libraries and efficient coding practices. I have learned to write clean, modular code and leverage Python’s extensive standard library for various applications. Transitioning to Flask, I started with the basics of creating simple web applications. Guided by senior developers, I delved into more complex features such as request handling, Jinja templating, and RESTful API development. I have gained experience in implementing authentication, database interactions using SQLAlchemy. This hands-on experience has not only solidified my understanding of Python and Flask but also provided me with practical insights into web development, enhancing my ability to create scalable, maintainable web applications.
    


\subsection{Result}
Upon successfully completing the training phase at BGSW Vietnam, I achieved a commendable score in the Codility Test, showcasing my adeptness in algorithmic problem-solving and technical proficiency. Additionally, I have acquired proficiency in utilizing a diverse array of tools and technologies essential to the software development lifecycle. \\

\noindent In summary, the comprehensive training program at BGSW Vietnam has been instrumental in augmenting my technical skills and equipping me with the capabilities to tackle complex software development challenges effectively. I am now poised to make substantial contributions to software engineering teams, drawing upon my expertise in programming, system design, and contemporary development methodologies. This experience has not only bolstered my technical acumen but also instilled confidence in navigating diverse and demanding project landscapes with competence and innovation.

% After the completion of my internship, I have fulfilled most of my expectations.