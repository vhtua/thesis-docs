\section{Literature Review}

\subsection{Existing solutions}

	\subsubsection{Group Chat-Based Systems (Current Solution at VGU)}
		Currently, many educational institutions, including VGU, rely on informal systems like social media group chats (e.g., Facebook or WhatsApp groups) for raising support tickets and contacting staff. While these systems are easy to set up and require minimal resources, they suffer from significant limitations:
		
		\begin{itemize}
			\item[-] \textbf{Lack of Structure}: The conversation threads are disorganized, making it hard to track specific issues or prioritize them.
			
			\item[-] \textbf{Absence of Accountability}: There’s no formal ticketing system, leading to delays in responses and no mechanism to track whether an issue has been resolved.
			
			\item[-] \textbf{Inadequate Historical Data}: It's difficult to retrieve past conversations or analyze data to improve service.
			
			\item[-] \textbf{Lack of Privacy}: Group chats often expose personal information to all participants, which may raise privacy concerns.
		\end{itemize}
		
		
		\subsubsection{Existing University Ticketing Systems}
		Several universities have adopted formal ticket management systems for handling student support services. These systems are often integrated into larger university management platforms or custom-built web applications. Common examples include:
		
		
		\begin{longtable}{{|m{4cm}|m{6cm}|m{6cm}|}} 
			\hline
			\textbf{Systems} & \textbf{Features} &\textbf{Limitations}\\ \hline
			JIRA Service Management & JIRA offers a robust ticket management system with features like customizable workflows, automated prioritization, and detailed tracking. Users can submit issues, track their status, and communicate with responsible parties. & JIRA is highly complex, designed primarily for enterprise IT systems, making it overkill for universities that require simpler workflows. It may also be costly and difficult to adapt to student needs without significant customization.
			\\ \hline
			Zendesk &  A widely-used customer support platform, Zendesk provides email, live chat, and ticketing systems. It also offers automation and analytics. & Zendesk’s focus is on customer service for businesses, so its workflows may not directly match the needs of a university setting, where student requests can be highly varied and require more personalized handling. It also lacks real-time communication features tailored for quick staff-student interactions.
			\\ \hline
			Freshdesk & Freshdesk includes support ticket management, knowledge base integration, and multi-channel communication (email, chat, etc.). It also supports collaboration between support agents. &  While Freshdesk is more user-friendly, it may still be too feature-heavy and expensive for small to medium-sized universities. Its primary market is businesses, so it lacks some specific educational features such as dormitory management or academic support.
			\\ \hline
			
			\caption{Existing University Ticketing Systems} % needs to go inside longtable environment
			\label{tab:existing-ticket-sys}
		\end{longtable}
		
	\subsubsection{Limitations of Existing Solutions in the University Context}
	
		\begin{itemize}
			\item[-] \textbf{Complexity}: Many existing solutions are designed for enterprise environments and are not tailored to the unique requirements of universities.
			\item[-] \textbf{Lack of Customization}: Solutions like JIRA and Zendesk require extensive customization to meet university-specific needs, such as handling dormitory issues or academic support tickets.
			\item[-] \textbf{Cost}: Proprietary solutions can be expensive, making them less viable for universities with limited IT budgets.
			\item[-]\textbf{ Lack of Real-Time Communication}: Most solutions offer asynchronous communication through email or message boards but do not provide real-time chat, which is essential for time-sensitive student support.
		\end{itemize}

\subsection{Technology Review}
	\subsection{This is subsection}
		
			