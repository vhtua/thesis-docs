\section{Literature Review}

\subsection{Existing solutions}

	\subsubsection{Group Chat-Based Systems (Current Solution at VGU)}
		Currently, many educational institutions, including VGU, rely on informal systems like social media group chats (e.g., Facebook or WhatsApp groups) for raising support tickets and contacting staff. While these systems are easy to set up and require minimal resources, they suffer from significant limitations:
		
		\begin{itemize}
			\item[-] \textbf{Lack of Structure}: The conversation threads are disorganized, making it hard to track specific issues or prioritize them.
			
			\item[-] \textbf{Absence of Accountability}: There’s no formal ticketing system, leading to delays in responses and no mechanism to track whether an issue has been resolved.
			
			\item[-] \textbf{Inadequate Historical Data}: It's difficult to retrieve past conversations or analyze data to improve service.
			
			\item[-] \textbf{Lack of Privacy}: Group chats often expose personal information to all participants, which may raise privacy concerns.
		\end{itemize}
		
		
		\subsubsection{Existing University and Open-source Ticketing Systems}
		Several universities have adopted formal ticket management systems for handling student support services. These systems are often integrated into larger university management platforms or custom-built web applications. Common examples include:
		
		\begin{longtable}{{|l|m{6cm}|m{6cm}|}} 
			\hline
			\textbf{Systems} & \textbf{Features} &\textbf{Limitations}\\ \hline
			\endhead
			JIRA Service Management 
			& Offers customizable workflows, automated prioritization, and detailed issue tracking.
			& Too complex for university needs, expensive, and difficult to adapt without major customization.
			\\ \hline
			Freshdesk 
			& Supports ticket management, multi-channel communication, and agent collaboration. 
			& Feature-heavy and expensive for universities; lacks educational-specific tools.
			\\ \hline
			Zendesk 
			&  Provides email, live chat, and ticketing, with automation and analytics. 
			& Geared towards businesses; lacks flexibility for diverse student needs and real-time communication.
			\\ \hline
			OSTicket
			& Open-source, customizable, with email-based ticketing and status tracking.
			& Requires customization for universities, not intuitive for non-technical users, lacks real-time communication.
			\\ \hline
			
			\caption{Existing University Ticketing Systems} % needs to go inside longtable environment
			\label{tab:existing-ticket-sys}
		\end{longtable}
		
	\subsubsection{Limitations of Existing Solutions in the University Context}
	
		\begin{itemize}
			\item[-] \textbf{Complexity}: Many existing solutions are designed for enterprise environments and are not tailored to the unique requirements of universities.
			\item[-] \textbf{Lack of Customization}: Solutions like JIRA and Zendesk require extensive customization to meet university-specific needs, such as handling dormitory issues or academic support tickets.
			\item[-] \textbf{Cost}: Proprietary solutions can be expensive, making them less viable for universities with limited IT budgets.
			\item[-]\textbf{ Lack of Real-Time Communication}: Most solutions offer asynchronous communication through email or message boards but do not provide real-time chat, which is essential for time-sensitive student support.
		\end{itemize}

\subsection{Technology Review}

	\subsubsection{Frontend: ReactJS, Material UI, Vite}
	
	
%	\begin{wrapfigure}{l}{0.25\textwidth}
%		\includegraphics[width=0.5\linewidth]{graphics/React_Logo_SVG.png} 
%		\caption{React}
%		\label{fig:react}
%	\end{wrapfigure}
	
	  \begin{tabular}{ @{} m{0.25\textwidth} m{0.7\textwidth} @{} }
		\begin{minipage}{\linewidth}
			\centering
			\includegraphics[width=0.6\linewidth]{graphics/React_Logo_SVG.png}
			\captionof{figure}{ReactJS Logo}
			\label{fig:react}
		\end{minipage}
		&
		\begin{minipage}{\linewidth}
			\textbf{ReactJS} is a popular JavaScript library for building user interfaces, which provides a fast, scalable, and modular way to develop the frontend of web applications \cite{react}. Its component-based architecture allows for reusability and efficient state management using hooks like \textbf{\texttt{useState()}} and \textbf{\texttt{useEffect()}}. This enables a responsive and dynamic user experience, ideal for handling real-time ticket updates.
		\end{minipage}
	\end{tabular}
	
	\vspace*{1cm}
	
	\begin{lstlisting}[language=Javascript, caption=Example of a React component]
		const Profile = () => {
			
			return (
				<MainCard title="Personal Information">
					<Grid container spacing={gridSpacing}>
				
						<Grid item xs={12} sm={6}>
							<ProfileCard />
						</Grid>
					
						<Grid item xs={12} sm={6}>
							<SchoolDetailsCard/>
						</Grid>
				
					</Grid>
				</MainCard>
			);
		}
		
		export default Profile;
	\end{lstlisting}
	
	\noindent With a vast array of libraries and tools available, React offers flexibility for adding extra functionality like routing, form handling, or animations. This helps build a rich, dynamic user experience.
	
	 \begin{tabular}{ @{} m{0.7\textwidth} m{0.25\textwidth} @{} }
		\begin{minipage}{\linewidth}
			\textbf{Material UI} is a React-based UI component library that implements Google’s Material Design principles. Material UI ensures that the frontend is both visually appealing and functionally intuitive. Pre-built components like buttons, forms, and dialogs accelerate development while maintaining consistency in design. \cite{material-ui} \\
			
			Material UI is a great option for creating responsive web applications with a sleek and contemporary design. It offers a vast selection of components that allow developers to efficiently build intricate layouts and user interfaces. Additionally, its well-organized and comprehensive API documentation makes it easy to explore each component and customize it to meet specific requirements. Whether you're developing a basic website or a more complex web application, Material UI helps you meet both design and functionality objectives while providing a seamless and responsive user experience. \cite{material-ui-why-use}
		\end{minipage}
		&
		\begin{minipage}{\linewidth}
				\centering
			\includegraphics[width=0.5\linewidth]{graphics/material-ui-480.png}
			\captionof{figure}{Material UI Logo}
			\label{fig:material-ui}

		\end{minipage}
	\end{tabular}

	
	\vspace*{0.5 cm}
	
	\begin{tabular}{ @{} m{0.25\textwidth} m{0.7\textwidth} @{} }
		\begin{minipage}{\linewidth}
			\centering
			\includegraphics[width=0.45\linewidth]{graphics/vite.png}
			\captionof{figure}{Vite Logo}
			\label{fig:vite}
		\end{minipage}
		&
		\begin{minipage}{\linewidth}			
			Vite is an innovative frontend build tool designed to enhance development speed and efficiency compared to traditional tools like Webpack. One of Vite's standout features is its instant hot module replacement (HMR), which significantly improves the developer experience by allowing changes to be reflected in the browser almost instantaneously without a full page reload. This feature is particularly beneficial during iterative development cycles, as it enables developers to see the effects of their code modifications in real-time, thereby accelerating the debugging process and fostering a more dynamic workflow. \cite{vite} 
			
			Furthermore, Vite leverages native ES modules in the browser, allowing for a more optimized development environment. Unlike older bundlers that require extensive preprocessing, Vite serves source files directly to the browser during development, resulting in quicker start-up times and faster builds. This approach not only enhances productivity but also simplifies the development setup, making it accessible for developers of all skill levels. \cite{why-vite}
			
%			Overall, Vite represents a significant advancement in frontend tooling, making it an ideal choice for modern web development, particularly for projects using frameworks like React. Its emphasis on speed, simplicity, and developer experience positions Vite as a compelling alternative to traditional build tools .
		\end{minipage}
	\end{tabular}
%	\vspace*{0.5 cm}
	
	
	\subsubsection{Backend: NodeJS, ExpressJS, SocketIO}
	
	\vspace*{0.5 cm}
	\begin{tabular}{ @{} m{0.25\textwidth} m{0.7\textwidth} @{} }
		\begin{minipage}{\linewidth}
			\centering
			\includegraphics[width=0.5\linewidth]{graphics/nodejs.png}
			\captionof{figure}{NodeJS Logo}
			\label{fig:nodejs}
		\end{minipage}
		&
		\begin{minipage}{\linewidth}
			\textbf{NodeJS} is a runtime that enables JavaScript to be used for server-side scripting, making it possible to use a single language (JavaScript) throughout the stack. NodeJS is non-blocking and event-driven, making it ideal for handling I/O-heavy tasks like managing support ticket requests in real time. \cite{nodejs-about}
			
			Node.js is also lightweight and efficient, allowing developers to use JavaScript on both the frontend and backend, which enhances flexibility and cross-functionality within teams, ultimately reducing development costs. This code reusability accelerates the development process, and since JavaScript is the most popular programming language, the single-thread event loop it employs makes asynchronous programming more manageable. Consequently, new engineers will find it easier to understand the application’s codebase. \cite{why-nodejs}
		\end{minipage}
	\end{tabular}
	
	\vspace*{0.8cm}
	
	\begin{tabular}{ @{} m{0.7\textwidth} m{0.25\textwidth} @{} }
		\begin{minipage}{\linewidth}
			\textbf{ExpressJS} is a minimalist web framework for NodeJS, Express simplifies routing, middleware management, and API handling. Over the years, ExpressJS has demonstrated significant scalability, evidenced by its widespread use among major companies operating it on their servers daily. It efficiently manages user requests and responses, needing minimal configuration for large-scale web application development. With its robust modules, packages, and resources, it supports developers in building reliable and scalable web applications.\cite{express-what-is} In this application, ExpressJS serves as the backbone of the server, processing requests from the frontend, interacting with the database, and managing the business logic.
		\end{minipage}
		&
		\begin{minipage}{\linewidth}
			\centering
			\includegraphics[width=0.8\linewidth]{graphics/expressjs.png}
			\captionof{figure}{Expressjs Logo}
			\label{fig:expressjs}
			
		\end{minipage}
	\end{tabular}
	
	\vspace*{0.5cm}
	\newpage
	\begin{tabular}{ @{} m{0.25\textwidth} m{0.7\textwidth} @{} }
		\begin{minipage}{\linewidth}
			\centering
			\includegraphics[width=0.45\linewidth]{graphics/socket-io.512x512.png}
			\captionof{figure}{SocketIO Logo}
			\label{fig:socketio}
		\end{minipage}
		&
		\begin{minipage}{\linewidth}
			\textbf{SocketIO} is a JavaScript library that enables real-time, bidirectional communication between clients and servers. \cite{socketio} In addition, Socket.IO offers various fallback options, such as long polling, \acs{jsonp} polling, and iframe-based transport. This allows for continued communication in environments where WebSocket is unavailable. \cite{socketio-vs-websocket} SocketIO is used to implement features such as real-time messaging between students and staff, making the system more interactive and responsive. 
		\end{minipage}
	\end{tabular}
	
	
	\subsubsection{Authentication: JWT, Redis}
	\vspace*{0.5cm}
	\begin{tabular}{ @{} m{0.25\textwidth} m{0.7\textwidth} @{} }
		\begin{minipage}{\linewidth}
			\centering
			\includegraphics[width=0.5\linewidth]{graphics/jwt-logo.png}
			\captionof{figure}{JWT Logo}
			\label{fig:jwt }
		\end{minipage}
		&
		\begin{minipage}{\linewidth}
			JWT (\acs{json} Web Tokens) is a token-based authentication system that provides secure stateless authentication for users. JWT is ideal for modern web applications because tokens can be stored on the client-side (in local storage or cookies) and are transmitted with each request, allowing for scalability. \cite{jwt}
		\end{minipage}
	\end{tabular}
	
	\begin{figure}[H]
		\centering
		\includegraphics[width=1.1\columnwidth]{graphics/jwt-explained.pdf}
		\caption{Detailed explanation of JWT-based authentication mechanism.}
		\label{fig:jwt-explained}
	\end{figure}
	
	
		\vspace*{0.5cm}
	\begin{tabular}{ @{} m{0.25\textwidth} m{0.7\textwidth} @{} }
		\begin{minipage}{\linewidth}
			\centering
			\includegraphics[width=0.5\linewidth]{graphics/redis.png}
			\captionof{figure}{Redis Logo}
			\label{fig:redis}
		\end{minipage}
		&
		\begin{minipage}{\linewidth}
			\textbf{Redis} is an in-memory data structure store, commonly used as a database, cache, and message broker. It is part of the NoSQL database category called key/value stores. Keys serve as unique identifiers, and their corresponding values can be any of the data types supported by Redis. These data types include basic Strings, Linked Lists, Sets, and Streams, each with its own specific behaviors and commands. \cite{redis-what-is} In this system, Redis is specifically utilized for session management, particularly in storing refresh tokens. By caching these tokens, Redis helps to reduce the load on the primary database, which improves both scalability and performance. Since Redis operates in memory, it allows for faster retrieval of session data, ensuring that the authentication process remains responsive and efficient. Additionally, Redis offers built-in features like automatic expiration, which helps in managing token lifetimes and ensuring secure token invalidation, further enhancing the security and reliability of the system.
		\end{minipage}
	\end{tabular}
	
	
	
	\subsubsection{Database: PostgreSQL}
	
	\begin{figure}[H]
		\centering
		\includegraphics[scale=0.1]{graphics/postgresql.png}
		\caption{PostgreSQL \acs{rdbms} Logo}
		\label{fig:postgresql}
	\end{figure}
	
	PostgreSQL is a powerful, open-source relational database that provides strong ACID (Atomicity, Consistency, Isolation, Durability) compliance, ensuring reliability and data integrity, which is critical for managing sensitive information such as user accounts, ticketing systems, and communication logs. Its robust support for advanced querying and indexing mechanisms ensures that the system can handle complex searches and queries with high efficiency, even as data grows over time. \cite{postgres}\\
	
	PostgreSQL also supports features like foreign key constraints, triggers, and stored procedures, which help in maintaining data consistency across multiple tables, ensuring that relationships between different data entities (such as users and tickets) are enforced and managed correctly. Additionally, its ability to support JSON and JSONB data types allows the system to store and query semi-structured data, offering flexibility in handling modern web applications that require both structured and unstructured data. \\
	
	The database also scales well, supporting a large number of concurrent users and high-volume transactions, making it an ideal choice for applications with growing user bases. PostgreSQL’s support for full-text search and geospatial data (via PostGIS) can be useful for implementing advanced search functionalities and geographical features in the system. With its extensibility (allowing the addition of custom functions, data types, and more), PostgreSQL provides a solid and scalable foundation for the back-end data management of the system. \\
	
	\subsubsection{Responsive Web Design: Techniques and Tools}
	
	\begin{itemize}
		\item \textbf{Media Queries}: CSS media queries are used to apply different styles based on device characteristics (screen size, resolution). This allows the frontend to automatically adapt to different devices, ensuring that the system is usable on desktops, laptops, tablets, and smartphones.
		
		\item \textbf{CSS Flexbox/Grid}: These CSS layout models allow for flexible, responsive layouts that adjust to different screen sizes. Flexbox is ideal for managing component positioning in small screens, while Grid is useful for creating complex layouts in larger screens.
	\end{itemize}


\subsection{Theoretical Background}

	\subsubsection{Ticket Management Systems}
	A ticket management system is a tool designed to manage and track the progress of support requests, from the time they are submitted until they are resolved. The system typically assigns a unique identifier (ticket) to each request, enabling staff to monitor progress, prioritize issues, and provide timely responses.
	In a university context, ticket management systems are particularly useful for handling student issues, such as dormitory problems, academic inquiries, and administrative requests. By assigning specific staff members to tickets, the system ensures accountability and reduces response time.
	
	\subsubsection{Real-Time Communication Tools}
	Real-time communication tools like SocketIO or WebSockets are essential in modern web applications. These tools allow for instantaneous data transmission between the server and client, enabling real-time messaging and live updates. For instance, in the Student Life Support Service, students and staff can exchange messages directly without having to refresh the page, ensuring efficient communication.
	
	\subsubsection{Web Application Development Best Practices}
	\begin{itemize}
		\item \textbf{Modular Design}: Applications should be developed in a modular fashion, separating concerns into distinct components (frontend, backend, database). This allows for easier maintenance and scalability.
		
		\item \textbf{Security First}: With the increasing number of security breaches in web applications, implementing security best practices like JWT for authentication, HTTPS for communication, and proper data validation is essential.
		
		\item \textbf{Responsive Design}: Ensuring that the application works across different devices and screen sizes is a fundamental best practice, especially for a university setting where students and staff might use a wide variety of devices.
	\end{itemize}
		
		
\subsection{Gap Analysis}

	\subsubsection{What is Missing from Existing Solutions}
	Existing solutions for university support systems face several shortcomings. Privacy concerns arise in social media-based group chats, where sensitive student information may be exposed, and even proprietary systems lack a strong focus on educational privacy needs. Role-specific functionalities are often missing, with few systems offering specialized tools for students, dormitory staff, or administrators, or including student-centric features like feedback collection, ticket rating, and public status views. Limited analytics is another issue; while general analytics are provided, they don’t cater to the specific needs of student services, such as tracking recurring issues or ticket performance. Additionally, many systems, like JIRA, are not user-friendly for students, requiring training and posing barriers in environments where simplicity is essential.
	
	\subsubsection{How the Student Life Support Service Fills These Gaps}
	The Student Life Support Service addresses the gaps in existing systems by offering a solution tailored specifically to university needs. Its customizable structure supports role-specific functionalities for students, dormitory staff, and administrators, making it ideal for managing university-specific scenarios like dormitory issues and academic inquiries. Real-time communication is enabled through SocketIO, allowing fast, interactive responses between students and staff. The user-friendly interface, built with ReactJS and Material UI, ensures easy navigation for non-technical users. As an open-source, cost-effective platform using NodeJS, PostgreSQL, and ReactJS, it avoids the high costs of proprietary software. The system also provides role-specific features, such as ticket creation, tracking, and rating for students, efficient ticket handling for staff, and detailed reporting tools for administrators. Enhanced privacy and security are ensured through JWT-based authentication and role-based access, preventing unauthorized access to sensitive information. Additionally, built-in data analytics offers administrators insights into ticket trends and areas for improvement in student support services.
	
	
	
	
	
			